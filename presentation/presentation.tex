\documentclass{beamer}

\mode<presentation> {

% The Beamer class comes with a number of default slide themes
% which change the colors and layouts of slides. Below this is a list
% of all the themes, uncomment each in turn to see what they look like.

%\usetheme{default}
%\usetheme{AnnArbor}
%\usetheme{Antibes}
%\usetheme{Bergen}
%\usetheme{Berkeley}
%\usetheme{Berlin}
%\usetheme{Boadilla}
%\usetheme{CambridgeUS}
%\usetheme{Copenhagen}
%\usetheme{Darmstadt}
%\usetheme{Dresden}
%\usetheme{Frankfurt}
%\usetheme{Goettingen}
%\usetheme{Hannover}
%\usetheme{Ilmenau}
%\usetheme{JuanLesPins}
%\usetheme{Luebeck}
\usetheme{Madrid}
%\usetheme{Malmoe}
%\usetheme{Marburg}
%\usetheme{Montpellier}
%\usetheme{PaloAlto}
%\usetheme{Pittsburgh}
%\usetheme{Rochester}
%\usetheme{Singapore}
%\usetheme{Szeged}
%\usetheme{Warsaw}

% As well as themes, the Beamer class has a number of color themes
% for any slide theme. Uncomment each of these in turn to see how it
% changes the colors of your current slide theme.

%\usecolortheme{albatross}
%\usecolortheme{beaver}
%\usecolortheme{beetle}
%\usecolortheme{crane}
%\usecolortheme{dolphin}
%\usecolortheme{dove}
%\usecolortheme{fly}
%\usecolortheme{lily}
%\usecolortheme{orchid}
%\usecolortheme{rose}
%\usecolortheme{seagull}
%\usecolortheme{seahorse}
%\usecolortheme{whale}
%\usecolortheme{wolverine}

%\setbeamertemplate{footline} % To remove the footer line in all slides uncomment this line
%\setbeamertemplate{footline}[page number] % To replace the footer line in all slides with a simple slide count uncomment this line

%\setbeamertemplate{navigation symbols}{} % To remove the navigation symbols from the bottom of all slides uncomment this line
}


\usepackage{lmodern}

\usepackage{scrextend}
%\changefontsizes{9pt}

\usepackage{graphicx}
\usepackage{subfig}

\usepackage{mathtools}
\usepackage{amsfonts}
\usepackage{bussproofs}
\usepackage{wrapfig}
\usepackage{graphicx} % Allows including images
\usepackage{booktabs} % Allows the use of \toprule, \midrule and \bottomrule in tables
\usepackage{bm} 
\usepackage{xspace}
\usepackage{setspace}
\usepackage{amsthm}
\usepackage{algorithm, algorithmic}
\newcommand{\RR}{{\sf I\kern-0.14emR}}
\DeclarePairedDelimiter\ceil{\lceil}{\rceil}
\DeclarePairedDelimiter\floor{\lfloor}{\rfloor}

%----------------------------------------------------------------------------------------
%	TITLE PAGE
%----------------------------------------------------------------------------------------

\title[Information Diffusion in Grids]{CS598HS Advanced Social \& Information Networks \\ Information Diffusion in Grid-like Social Networks} % The short title appears at the bottom of every slide, the full title is only on the title page

\author[Dong, Hill, Livanos]{Guangyu Dong, Randolph Hill, Vasileios Livanos} % Your name
\institute[UIUC] % Your institution as it will appear on the bottom of every slide, may be shorthand to save space
{
Department of Computer Science \\
University of Illinois at Urbana-Champaign \\ % Your institution for the title page
\medskip
\textit{gdong2@illinois.edu}, \textit{rwhill2@illinois.edu}, \textit{livanos3@illinois.edu} % Your email address
}
\date{December 13, 2017} % Date, can be changed to a custom date

\begin{document}

\begin{frame}
\titlepage % Print the title page as the first slide
\end{frame}


\section{Introduction}

\begin{frame}
\frametitle{Overview} % Table of contents slide, comment this block out to remove it
\tableofcontents[currentsection] % Throughout your presentation, if you choose to use \section{} and \subsection{} commands, these will automatically be printed on this slide as an overview of your presentation
\end{frame}

\begin{frame}{Problem Statement}

\pause
\begin{itemize}
\item We study how fast information propagates through variants of grid-like social networks.
\pause
\item We assume a fixed-weight, directed graph $G$ representing the social network, with nodes as agents and
edges that represent the interaction between two agents.
\pause
\item Each agent $i$ has a specific type $\theta_i \in [0,1]$, and interacts with another agent $j$ at a
time-varying frequency $f^*_{ij}(t)$.
\pause
\item In our model, the information can only be transmitted between agents of the same type, and that there
exists a source of information for each different agent type.
\pause
\item Finally, there is a constant probability $p$ that the information will leak through a single interaction.
This implies that an agent $j$ who interacts with frequency $f^*_{ij}(t)$ at time $t$ with another agent $i$ who has
the information, has a probability $p f^*_{ij}(t)$ of getting the information.
\end{itemize}

\end{frame}

\begin{frame}{Motivation}

\pause
\begin{itemize}
\item All people interact with other people in real social networks, and most of the time we procure information
through this interaction.
\pause
\item This information could be for example which individual is a good doctor, or a good car mechanic, and
procuring this information could drastically affect our quality of life.
\pause
\item Our problem is thus motivated by these examples, and we assume that there exist $k$ external service providers,
who provide services of different quality to agents of different type, but of the same quality to agents of the same type.
\pause
\item Thus, the information that is propagated through the network is assumed to be the identity of the highest quality
service provider for a specific type of agents.
\end{itemize}

\end{frame}

\begin{frame}{Preliminaries}

\pause
\begin{itemize}
\item Our analysis focuses on $3$ different graph structures:
	\pause
	\begin{enumerate}
	\item Plain Grid (PG): A $4$-regular graph where vertices are arranged in a grid pattern.
	\pause
	\item Unbiased Grid (UG): Similar to PG, but every node $u$ forms one additional edge with some other node
	$v$, which is picked randomly, with probability	proportional to $\frac{1}{d^2}$, where $d$ is
	the Manhattan distance of $u$ and $v$ in the grid.
	\pause
	\item Biased Grid (BG): Similar to UG but the probability of an agent $u$ forming an edge with another agent
	$v$ of the same type is proportional to $\frac{b}{d^2}$. $b$ is called the \textit{bias} of the graph.
	\end{enumerate}
\pause
\item We seek to compare the \textbf{Diffusion Rate (DR)} of information between these graphs, which is defined
as the expected time it will take for the information to spread to all agents in the graph who can potentially
learn it.
\pause
\item We seek to both develop a model and obtain some theoretical results about the DR between these graphs,
as well as perform several simulations that possibly confirm our theoretical analysis.
\end{itemize}

\end{frame}

\section{Theoretical Results}

\begin{frame}
\frametitle{Overview} % Table of contents slide, comment this block out to remove it
\tableofcontents[currentsection] % Throughout your presentation, if you choose to use \section{} and \subsection{} commands, these will automatically be printed on this slide as an overview of your presentation
\end{frame}

\begin{frame}{Plain Grid}

\pause
\begin{itemize}
\item We model our system as a \textbf{Time Inhomogeneous Markov Chain}. While TIMCs are extremely important,
not many theoretical results are known about them.
\pause
\item To make the problem tractable, we assume that any non-zero interaction between two agents is lower-bounded
by a constant $\delta > 0$, meaning that agents cannot have an arbitrarily small interaction.
\end{itemize}

\pause
\begin{block}{$C_t$: Possible Expansion Set}
\[
C_t = \left\{ \{i, j\} \in E(G) \mid i \in \mathcal{I}(t), j \in \mathcal{F}(t), \theta_i = \theta_j \right\}
\]
\end{block}

\pause
\begin{block}{Diffusion Rate}
For any graph $G$
\[
DR \leq n \sum_{t = 1}^{\infty} {\left( t \left( 1 - {(1 - \delta)}^{|C_t|} \right) \prod_{k = 1}^{t-1} {{(1 - \delta)}^{|C_k|} } \right) }
\]
\end{block}

\end{frame}

\begin{frame}{Unbiased and Biased Grids}

\pause
\begin{block}{Expected increase in $|C_t|$ in UG}
\[
\mathbb{E}\left[{|C^{UG}_t| - |C^{PG}_t|}\right] = \frac{ \left( \sum_{k = 2}^{2n} {\frac{1}{k^2}} \right) |\mathcal{I}(t)| \left( 1 - \frac{|\mathcal{I}(t)|}{n} \right) }{N_\theta}
\]
\end{block}

\pause
\begin{itemize}
\item We observe that the increase in DR between PG and UG is exponential!
\end{itemize}

\pause
\begin{block}{Expected increase in $|C_t|$ in BG}
\[
\mathbb{E}\left[{|C^{BG}_t| - |C^{PG}_t|} \right] = b \cdot \mathbb{E}\left[{|C^{UG}_t| - |C^{PG}_t|}\right]
\]
\end{block}

\pause
\begin{itemize}
\item Similarly, the increase in DR between BG and UG is exponential, and is dependent on the bias $b$.
\pause
\item Specifically, for $b = 2$ we expect the difference in DR between BG and UG to be larger than the
difference in DR between UG and PG.
\end{itemize}

\end{frame}

\section{Simulation Results}

\begin{frame}
\frametitle{Overview} % Table of contents slide, comment this block out to remove it
\tableofcontents[currentsection] % Throughout your presentation, if you choose to use \section{} and \subsection{} commands, these will automatically be printed on this slide as an overview of your presentation
\end{frame}

\begin{frame}{Diffusion Rate Simulations}

%\pause
%\begin{figure}
%        \centering
%        \begin{minipage}{.5\textwidth}
%            \centering
%            \includegraphics[width=.65\linewidth]{proof_technique.png}
%        \end{minipage}%
%        
%\end{figure}

\end{frame}

\begin{frame}

\begin{center}
\LARGE{QUESTIONS ?}
\end{center}

\begin{figure}
        \centering
        \begin{minipage}{.5\textwidth}
            \centering
            \includegraphics[width=.8\linewidth]{questions.png}
        \end{minipage}%
        
\end{figure}

\end{frame}

\end{document}
