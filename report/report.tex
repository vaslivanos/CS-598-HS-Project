\documentclass[A4paper,11pt]{article}

\usepackage{latexsym}
\usepackage{epsfig}
\usepackage{verbatim}
\usepackage{shadow}
\usepackage{amssymb}
\usepackage{amsmath}
\usepackage{graphicx}
\usepackage{color}
\usepackage{bm}
\usepackage{fancyhdr}

\addtolength{\textwidth}{4cm}
\addtolength{\textheight}{4cm}
\addtolength{\oddsidemargin}{-2cm}
\addtolength{\topmargin}{-2.5cm}

\newtheorem{theorem}{\sc Theorem}[section]
\newtheorem{lemma}[theorem]{\sc Lemma}
\newtheorem{corollary}[theorem]{\sc Corollary}
\newtheorem{fact}[theorem]{\sc Fact}
\newtheorem{remark}[theorem]{\sc Remark}
\newcommand{\Ex}{\mathop{\mathbb{E}}}

\author{
	{\sc Guangyu Dong} \\
	\texttt{gdong2@illinois.edu}
	\and
	{\sc Randolph Hill} \\
	\texttt{rwhill2@illinois.edu}
	\and
	{\sc Vasileios Livanos} \\
	\texttt{livanos3@illinois.edu}
}

\title{
Information Diffusion in Grid-like Social Networks
}

\date{}

\begin{document} \maketitle

\section{Introduction}

\par Consider $n$ agents that interact with each other in a social network. Each agent $i$ has a specific set of other agents
that they interact with, which is called the \textit{neighborhood} of $i$ and is denoted by $\mathcal{N}_i$. In this network,
each agent has a set of strategies available to them. Specifically, agent $i$ proposes an \textit{interaction frequency}
$f_{ij} \in \mathbb{R}$ to every agent $j \in \mathcal{N}_i$. Similarly, each agent $j$ proposes an
interaction frequency $f_{ji}$ to $i$. The frequency that they end up interacting at is simply the minimum of the two
proposals. However, in real social networks, the agents have different preferences for who to interact with. To capture this
in our model, we assign $w_{ij}$ to be the \textit{weight} $i$ places on $j$, or similarly how much $i$ values the interaction
with $j$. Further motivated by real social networks, we assign all agents a uniform ``budget" of interaction frequency $\beta$
that they cannot exceed.

\par In our social network, we assume that there exist some external \textit{service providers} that provide some service of a
certain quality to the agents. We assume a rather complicated framework, where each agent $i$ has a specific \textit{type},
denoted by $\theta_i \in [0, 1]$ and there exist $k$ distinct service providers that are not a part of the social network. Each
service provider has a different quality of service for each agent type, however the quality they provide is fixed for all agents
of the same type. This is again an attempt to model real-life social networks, where te quality of service an agent receives
is based on specific attributes they possess which in turn classify the agent as being of a certain type.

\par We make the following assumption. Each agent $i$ initially has a service provider chosen uniformly at random, and receives a
specific quality $q_i \in [0, 1]$ from them. Furthermore, $i$ gets some utility by communicating with all agents in their
neighborhood but also by learning about different service providers with higher quality than $q_i$ for their type. We assume
that, as $i$ communicates with a neighbor $j$ of the same type, for each unit of communication the probability that $i$ learns
$j$'s quality is $p$, which is constant and uniform for all agents in the network. Therefore, since between time $t$ and $t+1$
they interact $f^*_{ij}(t)$ times, the probability that $i$ learns $q_j$ is

\[
1 - {\left( 1 - p \right)}^{f^*_{ij}(t)}
\]

\par Also, since agents of different types exhibit different qualities from the same service provider, we make the assumption
that if $j$ is a neighbor of $i$, then $i$ is not interested in learning $q_j$ and therefore agents of type different than
$\theta_i$ contribute in $i$'s utility simply through their interaction and not by letting $i$ know about the existence of a
service provider with higher quality. Our final assumption is that once $i$ learns of a different service provider that offers
higher quality service to agents with type $\theta_i$, they immediately switch to that provider and start receiving the
aforementioned higher quality.

\par Thus, we model $i$'s utility by

\begin{equation}\label{eq:util}
u_i(t) = \sum_{j \in \mathcal{N}_i} {w_{ij} f^*_{ij}(t) \left( \beta - f^*_{ij}(t) \right) } +
\sum_{j \in \mathcal{Z}_i} {\Ex \left[ \min \left( q_j - q_i, 0 \right) \right] \left( 1 - {\left( 1 - p \right)}^{f^*_{ij}(t)} \right) }
\end{equation}

where $\mathcal{Z}_i$ is the subset of $i$'s neighbors that have the same type as $i$. Our model is sequential in that at each
time step $t$ one agent is chosen at random and updates their proposals to those that maximize their utility at time $t$.
This dynamics is called \textit{best-response}, since $i$ plays their best-response strategy to the strategies played by all
other players, assuming that they remain fixed at these strategies, at least for time $t$. It is easy to see that to play their
best-response strategy at time $t$, $i$ has to solve the following convex optimization problem for variables $f_{ij}(t)$

\begin{align}\label{eq:opt-prog}
\max & \: \: \: \sum_{j \in \mathcal{N}^i} {w_{ij} f^*_{ij}(t) \left( \beta - f^*_{ij}(t) \right) } +
\sum_{j \in \mathcal{Z}_i} {\Ex \left[ \min \left( q_j - q_i, 0 \right) \right] \left( 1 - {\left( 1 - p \right)}^{f^*_{ij}(t)} \right) } \\
s.t. & \: \: \: \sum_{j \in \mathcal{N}^i} {f_{ij}} \leq \beta \nonumber \\
& \: \: \: f_{ij} \geq 0, \: \: \forall j \in \mathcal{N}^i \nonumber \\
& \: \: \: f_{ij} \leq f_{ji}, \: \: \forall j \in \mathcal{N}^i \nonumber
\end{align}

\par While the solution to (\ref{eq:opt-prog}) is not trivial, we assume that all agents are able to solve it and to calculate
their best-response. Furthermore, it should be clear that since $p > 0$, given infinite time, all agents will eventually learn
the service provider that offers the highest quality for their type in their connected component, through their communication with
the agents of the same type. But this immediately raises another question; what is the rate of diffusion for this information and,
more importantly, how is it affected by -- seemingly -- small changes in the network structure? Our project's motivation is to
attempt to answer this question for three different grid-like graphs by analyzing the rate of information diffusion in each one.

\section{The plain $2$-dimensional grid}



%\label{Bibliography}

%\lhead{\emph{Bibliography}}

%\bibliographystyle{abbrv}

%\bibliography{report} % The references (bibliography) information are stored in the file named "Bibliography.bib"

\end{document}
