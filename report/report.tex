\documentclass[format=acmsmall, review=false]{acmart}

\usepackage{acm-ec-17}

\usepackage{booktabs} % For formal tables
\usepackage{latexsym}
\usepackage{epsfig}
\usepackage{verbatim}
\usepackage{shadow}
\usepackage{amssymb}
\usepackage{amsmath}
\usepackage{graphicx}
\usepackage{color}
\usepackage{bm}
\usepackage{fancyhdr}
\usepackage{hyperref}

%\newtheorem{theorem}{\sc Theorem}[section]
%\newtheorem{lemma}[theorem]{\sc Lemma}
%\newtheorem{corollary}[theorem]{\sc Corollary}
%\newtheorem{fact}[theorem]{\sc Fact}
%\newtheorem{remark}[theorem]{\sc Remark}
\newcommand{\Ex}{\mathop{\mathbb{E}}}

\usepackage[ruled]{algorithm2e} % For algorithms
\renewcommand{\algorithmcfname}{ALGORITHM}
\SetAlFnt{\small}
\SetAlCapFnt{\small}
\SetAlCapNameFnt{\small}
\SetAlCapHSkip{0pt}
\IncMargin{-\parindent}

\begin{document}

\title[Information Diffusion in Grid-like Social Networks]{Information Diffusion in Grid-like Social Networks}  
\author{Guangyu Dong}
\affiliation{%
  \institution{University of Illinois at Urbana-Champaign}
  \streetaddress{201 N Goodwin Street}
  \city{Urbana}
  \state{IL}
  \postcode{61801}
  \country{USA}
}
\author{Randolph Hill}
\affiliation{%
  \institution{University of Illinois at Urbana-Champaign}
  \streetaddress{201 N Goodwin Street}
  \city{Urbana}
  \state{IL}
  \postcode{61801}
  \country{USA}
}
\author{Vasileios Livanos}
\orcid{0000-0002-3425-0409}
\affiliation{%
  \institution{University of Illinois at Urbana-Champaign}
  \streetaddress{201 N Goodwin Street}
  \city{Urbana}
  \state{IL}
  \postcode{61801}
  \country{USA}
}

%\date{}

% note that the abstract must come before \maketitle
\begin{abstract}
\par In this paper we study how fast information propagates through various grid-like social networks. We
start by developing a model for information propagation in a general network and utilize time inhomogeneous
Markov chains to provide a lower bound on the \textit{diffusion rate} of information of any
general network. Through this model, we observe that as nodes in a $4$-regular grid graph start forming
new edges with other nodes, the diffusion rate of the network increases exponentially. Finally, we perform
several simulations and note that our initial results confirm our theoretical analysis.
\end{abstract}

% note: this command has been disabled to remove the ACM copyright block. Sorry...
%\thanks{This work is supported by the National Science Foundation, under grant CNS-0435060, grant CCR-0325197 and grant EN-CS-0329609.}

\maketitle


\section{Introduction}

\par In our lives, every single one of us is a part of several social networks. These social networks, besides
providing us with a means of communication with other individuals, perform another role; that of information
distribution. Indeed, through our interactions with other people we obtain several pieces of information every
day. This fact is actually extremely integral in our decision-making process, as we use this information to
make decisions that may potentially improve our quality of life. Examples of such phenomena include asking one's
friends and acquaintances who they believe to be the best doctor in a particular specialty, or who is their
current car mechanic and how satisfied they are with them. These kind of questions provide the motivation to
our work, which attempts to discern how fast this information can spread through a social network.

\par While this question is important in every setting, it is extremely difficult to analyze in the general case.
For that reason, we focus mainly on grid-like network structures, in an attempt to obtain more precise results.
Specifically, we consider a simple, $4$-regular graph as a basis, and compare the information's
\textit{diffusion rate} in this graph with that of two other graphs, variants of the plain grid. We note, however,
that our characterization of the diffusion rate of a graph -- for which a closed-form characterization was, to the
best of our knowledge, not known -- is defined for any network structure, which is an important step in getting
similar results in different, more general network structures.

\par Critical to the development of our model was the introduction of ideas from stochastic processes and ergodic
theory. Specifically, we formulate the problem as a question about expected hitting times of a time inhomogeneous
Markov chain, and use state-of-the-art techniques to develop novel theoretical results about this problem. While
time homogeneous Markov chains have been studied a lot in literature and many standard results about them are known,
their inhomogeneous variant has, as of yet, escaped most attempts made towards its analysis. Thus, not many results
about them are known, which in turn makes our results, albeit particularly model-specific, one of few such
developments in the field and adds to their significance.

\par Finally, we performed a series of simulations on these three network structures and oberved that, in most cases,
the simulation results seem to agree with the results predicted by our model. We take this to be at least a partial
confirmation that our approach is a correct and important step in understanding more about how information is
spread through a social network.


\section{Related Work}

\par This work was mainly inspired by novel approaches taken in analyzing how technologies diffuse in social networks
\cite{Immorlica}. While technology adoption can certainly be viewed as an information diffusion problem, the
constraints imposed on the model in most technology adoption problems make it difficult to obtain a general model that
is useful in real-life passive information diffusion. Information diffusion in social networks has of course been
studied in past literature, and there have been significant results, both theoretical
\cite{Jackson, Morris, Valente} and experimental \cite{Rogers, Cooper}. An active line of research in economics
and mathematical sociology is concerned with modeling these types of diffusion processes as a coordination game on
the social network \cite{Blume, Ellison, Jackson, Morris, Peyton}.

\par Another approach to information diffusion is the problem of \textit{influence maximization} where instead of
analyzing the passive diffusion of information in the social network, one attempts to compute the optimal set of
agents in the network which, if given the information, will trigger the maximum cascade of this information throughout
the network. This problem, while NP-hard in the general case \cite{Kempe}, has attracted a lot of attention and there
have been several recent attempts to utilize network structure in order to obtain strong theoretical guarantees
\cite{Seeman}.

\par On the stochastic side, time inhomogeneous Markov chains have evaded theoretical analysis for decades. However,
while early work in the field relied mostly on spectral techniques to provide guarantees on the asymptotic behavior
of such Markov chains \cite{Sonin, Fleischer}, more recent techniques utilize a combination of both spectral techniques
and state-of-the-art results from stability theory \cite{SC2, SC3, SC4, SC5} to provide a better understanding into
how non-homogeneous Markov chains behave. These results have very recently culminated in the development of highly
significant quantitative upper bounds on the expected mixing and hitting times of non-homogeneous Markov chains
\cite{SC1, Shen, Douc}, which we rely upon in our theoretical analysis.

\section{Theoretical Analysis}

\par In this section we present our main theoretical contributions. We start with some necessary definitions and background,
before we present our model which is based on time inhomogeneous Markov chains. We then use this model to derive a
characterization of the diffusion rate of information in any general graph. Finally, we specialize our model to several
variants of grid-like graphs and compare their diffusion rate.

\par Consider $n$ agents that interact with each other in a social network. Each agent $i$ has a specific set of other agents
that they interact with, which is called the \textit{neighborhood} of $i$ and is denoted by $\mathcal{N}_i$. In this network,
each agent has a set of strategies available to them. Specifically, agent $i$ proposes an \textit{interaction frequency}
$f_{ij} \in \mathbb{R}$ to every agent $j \in \mathcal{N}_i$. Similarly, each agent $j$ proposes an
interaction frequency $f_{ji}$ to $i$. The frequency $f^*_{ij} = f^*_{ji}$ that they end up interacting at is simply
\[
f^*_{ij} = \min { \{ f_{ij}, f_{ji} \} }
\]

\par However, in real social networks, the agents have different preferences for who to interact with. To capture this
in our model, we assign $w_{ij}$ to be the \textit{weight} $i$ places on $j$, or similarly how much $i$ values the interaction
with $j$. Further motivated by real social networks, we assign all agents a uniform ``budget" of interaction frequency $\beta$
that they cannot exceed.

\par In our social network, we assume that there exist some external \textit{service providers} that provide some service of a
certain quality to the agents. We assume a rather complicated framework, where each agent $i$ has a specific \textit{type},
denoted by $\theta_i \in [0, 1]$ out of $N_\theta$ possible types, and there exist $k$ distinct service providers that are not a
part of the social network. Each service provider has a different quality of service for each agent type, however the quality they
provide is fixed for all agents of the same type. This is again an attempt to model real-life social networks, where the quality
of service an agent receives is based on specific attributes they possess which in turn classify the agent as being of a certain
type. While we start byanalyzing the case where the agent types are drawn at random from the uniform distribution, we believe it
is of great importance to also study cases where neighboring agents have a greater chance of being of the same type. This implies
a clustering of agents' types, which is motivated by the homophily that real social networks exhibit in general.

\par We make the following assumption. Each agent $i$ initially has a service provider chosen uniformly at random, and receives a
specific quality $q_i \in [0, 1]$ from them. Furthermore, $i$ gets some utility by communicating with all agents in their
neighborhood but also by learning about different service providers with higher quality than $q_i$ for their type. We assume
that, as $i$ communicates with a neighbor $j$ of the same type, for each unit of communication the probability that $i$ learns
$j$'s quality is $p$, which is constant and uniform for all agents in the network. Therefore, since between time $t$ and $t+1$
they interact $f^*_{ij}(t)$ times, the probability that $i$ learns $q_j$ is
\[
1 - {\left( 1 - p \right)}^{f^*_{ij}(t)}
\]

\par Also, since agents of different type exhibit different qualities from the same service provider, we make the assumption
that if $j$ is a neighbor of $i$ and $\theta_i \neq \theta_j$, then $i$ is not interested in learning $q_j$ and therefore
agents of type different than $\theta_i$ contribute in $i$'s utility simply through their interaction and not by letting $i$
know about the existence of a service provider with higher quality. Our final assumption is that once $i$ learns of a different
service provider that offers higher quality service to agents with type $\theta_i$, they immediately switch to that provider
and start receiving the aforementioned higher quality.

\par Thus, we model $i$'s utility by
\begin{equation}\label{eq:util}
u_i(t) = \sum_{j \in \mathcal{N}_i} {w_{ij} f^*_{ij}(t) \left( \beta - f^*_{ij}(t) \right) } +
\sum_{j \in \mathcal{Z}_i} {\Ex \left[ \min \left( q_j - q_i, 0 \right) \right] \left( 1 - {\left( 1 - p \right)}^{f^*_{ij}(t)} \right) }
\end{equation}

where $\mathcal{Z}_i$ is the subset of $i$'s neighbors that have the same type as $i$. Our model is sequential in that at each
time step $t$ one agent is chosen at random and updates their proposals to those that maximize their utility at time $t$.
This dynamics is called \textit{best-response}, since $i$ plays their best-response strategy to the strategies played by all
other players, assuming that they remain fixed at these strategies, at least for time $t$. It is easy to see that to play their
best-response strategy at time $t$, $i$ has to solve the following convex optimization problem for variables $f_{ij}(t)$
\begin{align}\label{eq:opt-prog}
\max & \: \: \: \sum_{j \in \mathcal{N}^i} {w_{ij} f^*_{ij}(t) \left( \beta - f^*_{ij}(t) \right) } +
\sum_{j \in \mathcal{Z}_i} {\Ex \left[ \min \left( q_j - q_i, 0 \right) \right] \left( 1 - {\left( 1 - p \right)}^{f^*_{ij}(t)} \right) } \\
s.t. & \: \: \: \sum_{j \in \mathcal{N}^i} {f_{ij}} \leq \beta \nonumber \\
& \: \: \: f_{ij} \geq 0, \: \: \forall j \in \mathcal{N}_i \nonumber \\
& \: \: \: f_{ij} \leq f_{ji}, \: \: \forall j \in \mathcal{N}_i \nonumber
\end{align}

\par This problem will in general be very difficult to solve. In order to make it more tractable, we consider an approximation
to each agent's utility, where we replace
\[
1 - {\left( 1 - p \right)}^{f^*_{ij}(t)}
\]

with 
\[
p f^*_{ij}(t)
\]

This approximation makes sense only for $p << f^*_{ij}(t)$, therefore we limit our analysis to cases where $p$ is significantly
small, compared to the frequency proposals.

\par While the solution to \eqref{eq:opt-prog} is not trivial, we assume that all agents are able to solve it and to calculate
their best-response. Furthermore, it should be clear that since $p > 0$, given infinite time, all agents eventually learn
the service provider that offers the highest quality for their type in their connected component, through their communication with
the agents of the same type. But this immediately raises another question; what is the rate of diffusion for this information and,
more importantly, how is it affected by -- seemingly -- small changes in the network structure? Before we are ready to provide an
answer to this question, or even provide a clear definition of the diffusion rate in a graph, we have to introduce our Markov chain
model, which will make all such concepts significantly easier to analyze.

\subsection{Markov Chain Model}

\par For a specific time $t$, we define two sets of agents below
\[
\mathcal{I}(t) = \{ i \mid q_i \geq q_j \quad \forall j : \: \theta_j = \theta_i \}
\]
\[
\mathcal{F}(t) = \{ i \mid \exists j \in \mathcal{I}(t) \cap \mathcal{N}_i : \: \theta_j = \theta_i \land q_j > q_i \}
\]

\par It is understood that the agents in $\mathcal{I}(t)$ are the ones that possess the information at time $t$, while the agents
in $\mathcal{F}(t)$ are the agents which do not possess the information at time $t$, but have a neighbor that does. Equivalently,
$\mathcal{F}(t)$ is the set of agents to which the information may spread next. Next, we define a random variable $X_t$ which
represents the number of agents who have the service provider of the highest quality. In our model, we assume that $X_0 = 1$,
in the worst-case, and we want to calculate the information diffusion rate of $G$ defined below.

\begin{definition}[Diffusion Rate]
Consider a graph $G$ of $n$ agents. Then the \textbf{diffusion rate} of information in $G$ is
\[
\gamma_G = \frac{1}{\min_{t > 0} { \{ X_t = n \} }}
\]
\end{definition}

\par Consider a Markov chain on $X_t$ that is essentially a path graph of size $n$, enhanced with self-loops. Our initial state
is $X_0 = 1$ and, if we define the event $A^k_t$ as
\[
A^k_t = \{ X_t \geq k \mid X_{t-1} = k-1 \}
\]

for $k, t \geq 1$, then we would like to calculate the hitting time of the event $A^n_t$ which is the expected minimum $t$ for
which this event comes true for the first time. While normally we can analyze such hitting times on Markov chains with ease,
our inhomogeneous Markov chain provides an extra level of difficulty, due to the fact that the transition probabilities vary
with time. For that analysis, we need to define
\[
C_t = \left\{ \{ i, j\} \in E(G) \mid i \in \mathcal{I}(t), j \in \mathcal{F}(t), \theta_i = \theta_j \right\}
\]

as the set of all possible edges through which the information could leak through at time $t$.

\par Next, we present one of our main contributions, a lower bound on the information diffusion rate for any graph $G$,
or equivalently, an upper bound on the expected mixing time of our Markov chain on $G$

\begin{theorem}
Consider a graph $G$ of $n$ agents. Then, if
\[
\delta = \min_{t, \{ i, j \} \in C_t} {p f^*_{ij}(t)} > 0
\]

the diffusion rate of $G$ is
\[
\gamma_G \geq \frac{1}{ n \cdot \sum_{t = 1}^{\infty} {t \left( 1 - {\left( 1 - \delta \right)}^{|C_t|} \right) \cdot \prod_{\lambda = 1}^{t-1} { {\left( 1 - \delta \right)}^{|C_\lambda|} } } }
\]
\end{theorem}


\begin{proof}
\par We know that
\[
\Pr \left[ A^k_t \right] \geq 1 - \prod_{ \{i, j \} \in C_t} {\left( 1 - p f^*_{ij}(t) \right) }
\]

for any $k \geq 1$. We also know by \cite{SC1} that if we can lower bound the non-zero interactions between the agents by
a fixed constant $\delta > 0$, then our time inhomogeneous Markov chain converges almost surely to its (unique) stationary
distribution. Thus, for
\[
\delta = \min_{t, \{ i, j \} \in C_t} {p f^*_{ij}(t)}
\]

we get
\begin{equation}\label{probA}
\Pr \left[ A^k_t \right] \geq 1 - \prod_{ \{ i, j \} \in C_t } {\left( 1 - \delta \right) } = 1 - { \left( 1 - \delta \right) }^{|C_t|}
\end{equation}

\par Since we now know that our Markov chain converges, we can calculate the expected hitting time of $A^k_t$ for any
$k$
\begin{align}\label{hittime}
\mathbb{E}_t \left[ A^k_t \right] & = \sum_{t = 1}^{\infty} {t \Pr \left[ A^k_t \right] \cdot \prod_{\lambda = 1}^{t-1} {\left( 1 - \Pr \left[ A^k_\lambda \right] \right)} } \\ \nonumber
\mathbb{E}_t \left[ A^k_t \right] & = \sum_{t = 1}^{\infty} {t \left( 1 - {\left( 1 - \delta \right)}^{|C_t|} \right) \cdot \prod_{\lambda = 1}^{t-1} { {\left( 1 - \delta \right)}^{|C_\lambda|} } }
\end{align}

\par Finally, since the information can spread at most $n$ times, we know that there are only $n$ possible values of $k$
in $A^k_t$. Thus, the expected time it will take for the information to spread to all $n$ agents is
\[
\mathbb{E}_t \left[ A^n_t \right] \leq n \cdot \sum_{t = 1}^{\infty} {t \left( 1 - {\left( 1 - \delta \right)}^{|C_t|} \right) \cdot \prod_{\lambda = 1}^{t-1} { {\left( 1 - \delta \right)}^{|C_\lambda|} } }
\]

and we can provide a lower bound on the diffusion rate of $G$
\[
\gamma_G \geq \frac{1}{ n \cdot \sum_{t = 1}^{\infty} {t \left( 1 - {\left( 1 - \delta \right)}^{|C_t|} \right) \cdot \prod_{\lambda = 1}^{t-1} { {\left( 1 - \delta \right)}^{|C_\lambda|} } } } \qedhere
\]
\end{proof}

\subsection{Diffusion Rate in Grid-like Graphs}

\par Suppose $G$ is a $4$-regular graph of $n$ nodes where the vertex are arranged in a $2$-dimensional grid, and every edge
is either vertical or horizontal. We call this graph a \textit{plain grid}, and we use it as a basis for our analysis of its
variants. We denote the diffusion rate of the plain grid by $\gamma_{PG}$. Next, we define two variants of the plain grid
and compare their diffusion rates.

\subsubsection{\textbf{The unbiased grid}}

\par In the \textit{unbiased grid}, we extend the previous graph model by allowing each node $i$ to form one long-range edge
with some other node $j$ that does not immediately communicate with $i$ through the grid structure. However, the probability
distribution for this edge formation is not random, but instead is proportional to $\frac{1}{d^2}$, where $d$ is the Manhattan
distance, or equivalently the $L_1$ norm, defined as

\[
d(i, j) = |i.x - j.x| + |i.y - j.y|
\]

where $i.x$ denotes the row and $i.y$ denotes the column of agent $i$ in the grid. Note here that the edge formation
distribution does not depend on the types of agents, therefore, for sufficiently large $N_\theta$, we expect that most agents
will not form an edge with an agent of the same type. However, for those that will, the probability that they learn of a service
provider with higher quality for their type rises. We formalize this intuition with the following theorem

\begin{theorem}
Consider an instance $PG$ of the plain grid graph with $n$ nodes. Then, the expected increase of $|C_t|$ in the
corresponding unbiased grid $UG$ created from $PG$ is
\[
\mathbb{E} \left[ |C^{UG}_t| - |C^{PG}_t| \right] = \frac{\left( \sum_{k = 2}^{2n} {\frac{1}{k^2}} \right) |\mathcal{I}(t)| \left( 1 - \frac{|\mathcal{I}(t)|}{n} \right) }{N_\theta}
\]
\end{theorem}


\begin{proof}
\par The probability for a new edge $e = \{ i, j \}$ to be in $C_t$ is
\begin{equation}\label{probE}
\Pr \left[ e \in C_t \right] = \Pr \left[ \theta_i = \theta_j \right] \cdot \Pr \left[ j \notin \mathcal{I}(t) \mid \theta_i = \theta_j \right]
\end{equation}

\par Assuming the distribution of agents' types is uniform, we have
\[
\Pr \left[ \theta_i = \theta_j \right] = \frac{1}{N_\theta}
\]

and by using the fact that the events $\{ \theta_i = \theta_j \}$ and $\{ j \notin \mathcal{I}(t) \}$ are independent in
the unbiased grid, we get
\[
\Pr \left[ j \notin \mathcal{I}(t) \mid \theta_i = \theta_j \right] = \Pr \left[ j \notin \mathcal{I}(t) \right] = \left( \sum_{k = 2}^{2n} {\frac{1}{k^2}} \right) \left( 1 - \frac{|\mathcal{I}(t)|}{n} \right)
\]

\par Combining the above equalities, we get
\[
\Pr \left[ e \in C_t \right] = \frac{1}{N_\theta} \left( \sum_{k = 2}^{2n} {\frac{1}{k^2}} \right) \left( 1 - \frac{|\mathcal{I}(t)|}{n} \right)
\]

because $\Pr \left[ j \notin \mathcal{I}(t) \right]$ is calculated by fixing a new neighbor $j$ of $i$ (with probability
$\frac{1}{d^2}$ for $d \geq 2$), and then calculating the probability that $j \notin \mathcal{I}(t)$.

\par Therefore, we understand that the increase in $|C_t|$ on expectation is
\begin{equation}\label{incrUG}
\mathbb{E} \left[ |C^{UG}_t| - |C^{PG}_t| \right] = \frac{\left( \sum_{k = 2}^{2n} {\frac{1}{k^2}} \right) |\mathcal{I}(t)| \left( 1 - \frac{|\mathcal{I}(t)|}{n} \right) }{N_\theta} \qedhere
\end{equation}
\end{proof}

\par Note here that the diffusion rate $\gamma_G$ depends exponentially on $|C_t|$. Therefore, we understand that the increase
in diffusion rate between the unbiased and the plain grids is exponential. This fact alone is surprising, but we continue
by generalizing it to a larger family of graphs.

\subsubsection{\textbf{The biased grid}}

\par This previous observation naturally leads us to study a generalization of the previous graph model which we call the
\textit{biased grid}. In this graph structure, agents again form long-range edges to other agents in the grid with
probablities proportional to $\frac{1}{d^2}$, but they are biased towards forming an edge with an agent of the same type
as them. Therefore, we assume that, when an agent $i$ forms a long-range edge, agents with type $\theta_i$ are $b$ times
more likely to be chosen by $i$ than agents of different type. We call this parameter $b$, which directly affects the graph's
diffusion rate, the \textit{bias} of $G$.

\par In the biased case, the probability that an edge $e$ is in $C_t$ changes, as the events $\{ \theta_i = \theta_j \}$
and $\{ j \notin \mathcal{I}(t) \}$ are not independent anymore. It turns out that the expected increase in the diffusion
rate of the biased grid compared to the unbiased grid is again exponential and behaves in a nice linear fashion
depending on the bias $b$, as is demonstrated by the following theorem

\begin{theorem}
Consider an instance $PG$ of the plain grid graph with $n$ nodes. Then, the expected increase of $|C_t|$ in the
corresponding biased grid $BG$ with bias $b$ created from $PG$ is
\[
\mathbb{E} \left[ |C^{BG}_t| - |C^{PG}_t| \right] = \frac{b \cdot \left( \sum_{k = 2}^{2n} {\frac{1}{k^2}} \right) |\mathcal{I}(t)| \left( 1 - \frac{|\mathcal{I}(t)|}{n} \right) }{N_\theta} = b \cdot \mathbb{E} \left[ |C^{UG}_t| - |C^{PG}_t| \right] 
\]
\end{theorem}


\begin{proof}
\par As stated above, the events $\{ \theta_i = \theta_j \}$ and $\{ j \notin \mathcal{I}(t) \}$ are not independent
anymore, thus the probability that an edge $e$ is in $C_t$ changes. We know that
\[
\Pr \left[ \{i, j \} \text{is formed} \right] = \begin{cases}
\frac{b}{d^2}, & \theta_i = \theta_j \\ 
\frac{1}{d^2}, & \theta_i \neq \theta_j
\end{cases}
\]

\par Utilizing the above equation, we can write
\[
\Pr \left[ e \in C_t \right] = \frac{1}{N_\theta} \left( \sum_{k = 2}^{2n} {\frac{b}{k^2}} \right) \left( 1 - \frac{|\mathcal{I}(t)|}{n} \right)
\]

\par Thus

\[
\mathbb{E} \left[ |C^{BG}_t| - |C^{PG}_t| \right] = \frac{b \cdot \left( \sum_{k = 2}^{2n} {\frac{1}{k^2}} \right) |\mathcal{I}(t)| \left( 1 - \frac{|\mathcal{I}(t)|}{n} \right) }{N_\theta} = b \cdot \mathbb{E} \left[ |C^{UG}_t| - |C^{PG}_t| \right] \qedhere
\]
\end{proof}

\par We observe that, once again, the increase between the biased and the plain grids in $|C_t|$ -- thus also in $\gamma_G$
-- is exponential. However, the increase in diffusion rate between the biased and unbiased grids is also exponential!
Specifically, for $b = 2$, we have
\[
\mathbb{E} \left[ |C^{BG}_t| - |C^{UG}_t| \right] = \mathbb{E} \left[ |C^{UG}_t| - |C^{PG}_t| \right]
\]

thus (because $\gamma_G$ depends exponentially on $|C_t|$), if we observe an increase of $a$ in the diffusion rate between
the unbiased and plain grids, we expect to see an increase of $a^2$ (or $a^b$ in general) between the biased and plain grids.
Equivalently, we expect to see an increase of $a$ (or $a^{b-1}$ in general) between the biased and unbiased grids.


\section{Simulations}

\section{Conclusion}

%\begin{acks}
%	
%	The authors would like to thank Robert Andrews from the University of
%	Illinois at Urbana-Champaign for providing critical insight on the
%	theoretical analysis of our model.
%	
%\end{acks}

% Bibliography
\bibliographystyle{ACM-Reference-Format}
\bibliography{report}

\end{document}
