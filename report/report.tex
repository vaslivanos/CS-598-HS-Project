\documentclass[A4paper,11pt]{article}

\usepackage{latexsym}
\usepackage{epsfig}
\usepackage{verbatim}
\usepackage{shadow}
\usepackage{amssymb}
\usepackage{amsmath}
\usepackage{graphicx}
\usepackage{color}
\usepackage{bm}
\usepackage{fancyhdr}

\addtolength{\textwidth}{4cm}
\addtolength{\textheight}{4cm}
\addtolength{\oddsidemargin}{-2cm}
\addtolength{\topmargin}{-2.5cm}

\newtheorem{theorem}{\sc Theorem}[section]
\newtheorem{lemma}[theorem]{\sc Lemma}
\newtheorem{corollary}[theorem]{\sc Corollary}
\newtheorem{fact}[theorem]{\sc Fact}
\newtheorem{remark}[theorem]{\sc Remark}
\newcommand{\Ex}{\mathop{\mathbb{E}}}

\author{
	{\sc Guangyu Dong} \\
	\texttt{gdong2@illinois.edu}
	\and
	{\sc Randolph Hill} \\
	\texttt{rwhill2@illinois.edu}
	\and
	{\sc Vasileios Livanos} \\
	\texttt{livanos3@illinois.edu}
}

\title{
Information Diffusion in Grid-like Social Networks
}

\date{}

\begin{document} \maketitle

\section{Introduction}

\par Consider $n$ agents that interact with each other in a social network. Each agent $i$ has a specific set of other agents
that they interact with, which is called the \textit{neighborhood} of $i$ and is denoted by $\mathcal{N}_i$. In this network,
each agent has a set of strategies available to them. Specifically, agent $i$ proposes an \textit{interaction frequency}
$f_{ij} \in \mathbb{R}$ to every agent $j \in \mathcal{N}_i$. Similarly, each agent $j$ proposes an
interaction frequency $f_{ji}$ to $i$. The frequency that they end up interacting at is simply the minimum of the two
proposals. However, in real social networks, the agents have different preferences for who to interact with. To capture this
in our model, we assign $w_{ij}$ to be the \textit{weight} $i$ places on $j$, or similarly how much $i$ values the interaction
with $j$. Further motivated by real social networks, we assign all agents a uniform ``budget" of interaction frequency $\beta$
that they cannot exceed.

\par In our social network, we assume that there exist some external \textit{service providers} that provide some service of a
certain quality to the agents. We assume a rather complicated framework, where each agent $i$ has a specific \textit{type},
denoted by $\theta_i \in [0, 1]$ and there exist $k$ distinct service providers that are not a part of the social network. Each
service provider has a different quality of service for each agent type, however the quality they provide is fixed for all agents
of the same type. This is again an attempt to model real-life social networks, where te quality of service an agent receives
is based on specific attributes they possess which in turn classify the agent as being of a certain type. While we start by
analyzing the case where the agent types are drawn at random from the uniform distribution, we believe it is of great
importance to also study cases where neighboring agents have a greater chance of being of the same type. This implies a
clustering of agents' types, which is motivated by the homophily that real social networks exhibit in general.

\par We make the following assumption. Each agent $i$ initially has a service provider chosen uniformly at random, and receives a
specific quality $q_i \in [0, 1]$ from them. Furthermore, $i$ gets some utility by communicating with all agents in their
neighborhood but also by learning about different service providers with higher quality than $q_i$ for their type. We assume
that, as $i$ communicates with a neighbor $j$ of the same type, for each unit of communication the probability that $i$ learns
$j$'s quality is $p$, which is constant and uniform for all agents in the network. Therefore, since between time $t$ and $t+1$
they interact $f^*_{ij}(t)$ times, the probability that $i$ learns $q_j$ is

\[
1 - {\left( 1 - p \right)}^{f^*_{ij}(t)}
\]

\par Also, since agents of different types exhibit different qualities from the same service provider, we make the assumption
that if $j$ is a neighbor of $i$, then $i$ is not interested in learning $q_j$ and therefore agents of type different than
$\theta_i$ contribute in $i$'s utility simply through their interaction and not by letting $i$ know about the existence of a
service provider with higher quality. Our final assumption is that once $i$ learns of a different service provider that offers
higher quality service to agents with type $\theta_i$, they immediately switch to that provider and start receiving the
aforementioned higher quality.

\par Thus, we model $i$'s utility by

\begin{equation}\label{eq:util}
u_i(t) = \sum_{j \in \mathcal{N}_i} {w_{ij} f^*_{ij}(t) \left( \beta - f^*_{ij}(t) \right) } +
\sum_{j \in \mathcal{Z}_i} {\Ex \left[ \min \left( q_j - q_i, 0 \right) \right] \left( 1 - {\left( 1 - p \right)}^{f^*_{ij}(t)} \right) }
\end{equation}

where $\mathcal{Z}_i$ is the subset of $i$'s neighbors that have the same type as $i$. Our model is sequential in that at each
time step $t$ one agent is chosen at random and updates their proposals to those that maximize their utility at time $t$.
This dynamics is called \textit{best-response}, since $i$ plays their best-response strategy to the strategies played by all
other players, assuming that they remain fixed at these strategies, at least for time $t$. It is easy to see that to play their
best-response strategy at time $t$, $i$ has to solve the following convex optimization problem for variables $f_{ij}(t)$

\begin{align}\label{eq:opt-prog}
\max & \: \: \: \sum_{j \in \mathcal{N}^i} {w_{ij} f^*_{ij}(t) \left( \beta - f^*_{ij}(t) \right) } +
\sum_{j \in \mathcal{Z}_i} {\Ex \left[ \min \left( q_j - q_i, 0 \right) \right] \left( 1 - {\left( 1 - p \right)}^{f^*_{ij}(t)} \right) } \\
s.t. & \: \: \: \sum_{j \in \mathcal{N}^i} {f_{ij}} \leq \beta \nonumber \\
& \: \: \: f_{ij} \geq 0, \: \: \forall j \in \mathcal{N}^i \nonumber \\
& \: \: \: f_{ij} \leq f_{ji}, \: \: \forall j \in \mathcal{N}^i \nonumber
\end{align}

\par This problem will in general be very difficult to solve. In order to make it more tractable, we consider an approximation
to each agent's utility, where we replace

\[
1 - {\left( 1 - p \right)}^{f^*_{ij}(t)}
\]

with 

\[
p f^*_{ij}(t)
\]

This approximation makes sense only for $p << f^*_{ij}(t)$, therefore we limit our analysis to cases where $p$ is significantly
small.

\par While the solution to (\ref{eq:opt-prog}) is not trivial, we assume that all agents are able to solve it and to calculate
their best-response. Furthermore, it should be clear that since $p > 0$, given infinite time, all agents will eventually learn
the service provider that offers the highest quality for their type in their connected component, through their communication with
the agents of the same type. But this immediately raises another question; what is the rate of diffusion for this information and,
more importantly, how is it affected by -- seemingly -- small changes in the network structure? Our project's motivation is to
attempt to answer this question for three different grid-like graphs by analyzing the rate of information diffusion in each one.

\subsection{The plain $2$-dimensional grid}

\par Consider a graph $G = (V, E)$ where the vertex are arranged in a $2$-dimensional grid, and every edge is either vertical
or horizontal. We are interested in calculating the rate of information diffusion in this graph. Suppose that all agents are of
the same type, $\theta$, and there exists only one agent that has the service provider with the highest quality, without loss of
generality, let that agent be the one represented by the upper left corner vertex in the grid. We understand that this is the
worst-case scenario for this graph, because more agent types would imply that the information diffusion analysis splits to many
smaller graphs, and if more agents have the service provider with the highest quality then the probability that this information
spreads rises. In both cases, the rate of information diffusion would be faster.

\par We use this graph as a starting point for our analysis, because it simplifies our model, while being the worst-case scenario
for this graph structure. However, later we move on to the general case with multiple agent types and agents who have the
service provider of the highest quality.

\subsection{The extended $2$-dimensional grid}

\par Next, we extend the previous graph model by allowing each node $i$ to form one long-range edge with some other node $j$ that
does not immediately communicate with $i$ through the grid structure. However, the probability distribution for this edge
formation is not random, but instead is proportional to $\frac{1}{d^2}$, where $d$ is the Manhattan distance, or equivalently
the $L_1$ norm, defined as

\[
d(i, j) = |i.x - j.x| + |i.y - j.y|
\]

where $i.x$ denotes the row and $i.y$ denotes the column of agent $i$ in the grid. Note here that the edge formation
distribution does not depend on the types of agents, therefore we expect that most agents will not form an edge with
an agent of the same type. However, for those that will, the probability that they learn of a service provider with higher
quality for their type rises; albeit only by a small fraction. Thus, when we calculate the rate of information diffusion for
this graph, we expect it to be smaller than that of the plain $2$-d grid, but not by a significant factor.

\subsection{The biased $2$-dimensional grid}

\par This previous observation naturally leads us to study a small variation of the previous graph model, namely the biased
$2$-d grid. In this model, agents again form long-range edges to other agents in the grid with probablities proportional
to $\frac{1}{d^2}$, but they are biased towards forming an edge with an agent of the same type as them. Therefore, we
assume that, when an agent $i$ forms a long-range edge, agents with type $\theta_i$ are $c$ times more likely to be chosen
by $i$ than agents of different type.

\par We plan on varying the parameter $c$ to observe how the rate of information diffusion behaves, compared to the two
previous models, starting from $c = 2$. Intuitively, we expect this network structure to exhibit a very significant
difference in the rate of information diffusion for $c \geq c_{cr}$. Our main goal with this network structure is to
calculate this $c_{cr}$. The results we get may hopefully reveal why real-life social networks exhibit homophily and
arrange themselves in type clusters.


\section{Initial Results}

\par In this final section, we provide some initial results of our theoretical analysis. Hopefully, the ideas presented
here will function as a basis for our main goals of this problem. For a specific time $t$, we define two sets of agents below

\[
\mathcal{I}(t) = \{ i : i \: \: \text{has the service provider of the highest quality.} \}
\]

\[
\mathcal{F}(t) = \{ i : i \: \: \text{does not have the service provider of the highest quality, but some} \: \: j \in \mathcal{N}_i \: \: \text{does.} \}
\]

\par It is understood that the agents in $\mathcal{I}(t)$ are the ones who know the information at time $t$, while the agents
in $\mathcal{F}(t)$ are the agents to which the information will spread next. Next, we define a random variable $X_t$ which
represents the number of agents who have the service provider of the highest quality. Therefore, in our worst-case model, we
assume $X_0 = 1$ and we want to calculate

\[
\lambda^* = \min_{\lambda > 0} { \{ X_\lambda = n \} }
\]

\par We now present our main contribution so far. Consider a Markov chain on $X_t$ that is essentially a path graph of size $n$,
enhanced with self-loops. Our initial state is $X_0 = 1$ and we would like to calculate the hitting time of the event
$\{ X_t \geq n \}$; meaning the expected minimum $t$ for which this event comes true. While normally we can analyze such
hitting times on Markov chains with ease, our Markov chain has an extra level of difficulty, due to the fact that the transition
probabilities vary with time. Let $F = |\mathcal{F}(t)|$, and consider an agent $i \in \mathcal{I}(t)$ and an agent
$j \in \mathcal{F}(t)$. Then, at time $t$, we get

\[
Pr[X_{t+1} = k + z | X_t = k] = \binom{F}{z} { \left( p f^*_{ij}(t) \right) }^z { \left( 1 - p f^*_{ij}(t) \right) }^{F-z}
\]

\par Such Markov chains with time-varying transition probabilities are called \textit{time inhomogeneous Markov chains}, and
while there have been several attempts to argue about their convergence properties and analyze them, to the best of our
knowledge no significant convergence results are known them. We hope our analysis for this project sheds light both to
these models discussed earlier, but also provides new tools that contribute to the analysis of generic time-inhomogeneous
Markov chains.

\par Finally, we plan on performing several simulations that compare the rate of diffusion in the aforementioned models,
varying many paramteres to see their effect on the network. Our hope is that these simulations confirm the theoretical
results that we develop, or provide some insight that highlights the limitations of our theoretical analysis.

%\label{Bibliography}

%\lhead{\emph{Bibliography}}

%\bibliographystyle{abbrv}

%\bibliography{report} % The references (bibliography) information are stored in the file named "Bibliography.bib"

\end{document}
